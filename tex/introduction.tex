Mind wandering (MW) is a phenomenon where the thoughts of a person shift from task-related to task-unrelated for a certain period of time. Every person has probably experienced it at some point and it can come paired with inefficiency and distraction. In some situations where attention is needed, such as driving, it is important that the mind wanders as little as possible so people can focus on the task at hand. In \cite{berthie2015restless}, it is stated that recent research has clearly shown that inattention when driving has an indisputable impact on road safety. It is clear that automated detection and prevention of mind wandering in such settings could be a good solution to the problem. The main point in this paper is to give a structured overview of the state-of-the-art methods in the field of MW detection and the performance that these methods get. This study could be used to identify possible challenges in the field in order to further develop it, or as a starting point for people who wish to undertake research of their own and want an overview of the possibilities.