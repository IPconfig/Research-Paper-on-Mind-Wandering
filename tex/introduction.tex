Mind wandering (MW) is a phenomenon where the thoughts of a person shift from task-related to task-unrelated for a certain period of time.
Every person has probably experienced it at some point and it can come paired with inefficiency and distraction. 
In situations where attention is needed, such as driving, it is important that the mind wanders as little as possible so people can focus on the task at hand. 
Recent research has clearly shown that inattention when driving has an indisputable impact on road safety \cite{berthie2015restless}. 
It is clear that automated detection and prevention of mind wandering in such settings could be a good solution to the problem. 
This study aims to identify possible challenges in the field of automatic detection of MW in order to further develop it or to be used as a starting point for people who wish to do research and want an overview of the possibilities and what has been done already.
This goal is accomplished by giving a structured overview of the state-of-the-art methods in the field and reasoning about the broader context of these studies.

First of all, it is questioned in what tasks the automatic detection of MW is being utilized. 

Second, it is questioned what modalities are used, which sensors are used for recording these modalities and what features are extracted from them. 

Third, it is evaluated how MW is being reported by the participants. 

Fourth, it is checked what machine learning (ML) algorithms are used to train the model, and what performance is achieved by the model.

Last, it is questioned in which environment the data has been collected and if this data has been made publicly available.