All studies were performed in some kind of experimental setting where participants are unfamiliar with their surroundings and/or equipment. 
Personal experience and the current environment determines how much \emph{interference effects} a distractor will generate \cite{Roda2006AttentionAgenda}. This interference effect explains why it is not always possible to keep your attention on a target.
Since participants are in an unfamiliar environment and do not use their own equipment, they could experience \emph{more} interference than normal. 
Contrariwise, because participants know their performance is being measured, it could also be that they are more focused, leading to \emph{less} interference.
The amount of interference could influence when, and even if a participant self-report their mind wandering. This has a big effect on the training of ML models and their results.
It would be interesting to perform the studies in a more natural environment to see the effects this interference could have.