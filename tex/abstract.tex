Objective: The aim is to give a clear and structured overview of the current state of the art of automatic detection of mind wandering from multimodal datastreams.

Method: This survey is a literature study done by finding and analyzing existing publications regarding the subject. From an initial 125 publications, a total of 18 were selected to be analyzed in our study.

Results: All evaluated studies are summarized in Table \ref{tab:data}, including the performance of the study if applicable.

Conclusion: Considering how new the automatic detection of mind wandering is as a research subject, there have been a lot of studies using different detection technologies with good success. This shows that there are good opportunities for real-world applications of these technologies that can help prevent mind wandering, for example to prevent mind wandering while driving or when attending a lecture.