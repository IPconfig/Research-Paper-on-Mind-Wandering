This study aims to give a clear and structured overview of the current state-of-the-art of automatic detection of mind wandering from multimodal datastreams. This survey is a literature study done by finding and analyzing existing studies regarding the subject. From an initial 72 studies, a total of 18 were selected to be analyzed in our study. A technical overview of all evaluated studies is given in Table \ref{tab:data}. This table, alongside the context of the studies, is then further analyzed in the corresponding text. Considering how new the automatic detection of mind wandering is as a research subject, there have been a lot of studies using different detection technologies with good success. This shows that there are good opportunities for real-world applications of these technologies that can help prevent mind wandering. However, it would be beneficial for future research to have standardized reporting methods and datasets such that performance comparisons are easier to make.