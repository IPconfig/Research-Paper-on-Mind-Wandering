A modality is a channel of information provided by a sensor. For example, the modality physiology is used in \cite{Blanchard2014AutomatedLearning}. This means that the study used some features of physiology to detect mind wandering. In this case, the study looked at the skin conductance and the skin temperature of the test subject. There are a lot of different modalities used in the studies that were looked at. A few modalities that were used are words in videos, pupillometry and electroencephalographic brain activity. The most common modality used in these studies is eye gaze. This could be because there are a lot of ways in which mind wandering could be correlated to eye gaze, since eye gaze has already shown promising results in various studies. It is also one of the easiest things to track in terms of equipment, seeing as you can do it with a simple webcam, unlike modalities like brain activity, which require expensive EEG scans and such.

To be able to do the research, the researchers have to extract features from the modalities. In Table \ref{tab:data}, the features extracted from modalities are listed. The abbreviations of the features can be found underneath the table. Every study was precise in describing the features that were extracted. Exactly specifying which features are extracted gives a better overview of what features might be important for the detection of MW. In \cite{Hutt2017OutClassroom} it is stated that 57 global gaze features were extracted. Global gaze feature are independent of what people are looking at, while local gaze features do take this into account. Features that are extracted the most when looking at physiology are skin conductance and skin temperature. Other features that are often used are context features. These do not belong to a modality that is recorder by a sensor. These features are related to the task the participant has to perform and can differ from page length to task difficulty.