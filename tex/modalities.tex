A modality is an attribute or circumstance that denotes in what manner something is done or executed. For example, the modality physiology is used in \cite{Blanchard2014AutomatedLearning}. This means that the study used some form of physiology to detect mind wandering. In this case the study looked at the skin conductance and the skin temperature of the test subject. There are a lot of different modalities used in the studies that were looked at. A few modalities that were used are words in videos, pupillometry and electroencephalographic brain activity. The most common modality used in these studies is eye gaze. There is no real reason as to why eye gaze is the most used modality. One might say it is because there are a lot of ways in which mind wandering could be correlated to eye gaze. Also, eye gaze has already shown promising results.

To be able to do the research, the researchers have to extract features from the modalities. In Table 1, the features extracted from modalities are listed. The abbreviations of the features can be found below the table. Every study was precise in describing the features that were extracted. In \cite{Hutt2017OutClassroom} it is stated that 57 global gaze features were extracted. Global gaze feature are independent of what people are looking at, while local gaze features do take this into account. Exactly specifying which features are extracted gives a better overview of what features might be important for the detection of MW.