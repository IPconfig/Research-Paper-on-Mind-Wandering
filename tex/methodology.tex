
In order to identify all articles that may be deemed relevant, a highly systematic approach was used that can be broken down into six steps (Fig. \ref{fig:prisma}).

First, 10 articles were hand-picked that were identified as relevant for our goal. These hand-picked articles served as a baseline for the upcoming search of online libraries.
A systematic search on Scopus and Web of Science was performed to identify all articles aimed at the detection of mind wandering. The retrieved articles were checked on relevancy and completeness.
An indication of the completeness of the search terms used was obtained by comparing the results of the online libraries with the baseline. 
The used search terms were tweaked by making them broader when the results contained a low number of articles and few baseline articles and made narrower when there were many irrelevant articles retrieved.
A logbook of the used search terms can be found in Appendix A.
As a result of this first step, an initial list of 125 articles was obtained.

\begin{figure}
  \resizebox{\columnwidth}{!}{\begin{tikzpicture}[
    node distance=15mm and 10mm,
    start chain=going below,
 mynode/.style = {
        draw, rectangle, align=center, text width=60mm,
        font=\large, inner sep=1ex, outer sep=0pt,
        on chain},
mylabel/.style = {
        draw, rectangle, align=center, rounded corners, 
        font=\small\bfseries, inner sep=2ex, outer sep=0pt,
        fill=cyan!30, minimum height=35mm,
        on chain},
every join/.style = arrow,
     arrow/.style = {very thick,-stealth}
                    ] 
\coordinate (tc);
% the title
%\node[above=of tc,font=\bfseries] {PRISMA 2009 Flow Diagram};
% the nodes at the top
\node (n1a) [mynode, left=of tc, xshift=7mm]    {Studies identified through
                                        database searching\\
                                        (N = 125)};
\node (n1b) [mynode,right=of tc, xshift=-7mm]    {Additional Studies indentified\\
                                        through other sources\\
                                        (N = 10)};
    % the chain in the center
\node (n2)  [mynode, below=of tc]   {Studies after duplicates removed\\
                                        (N = 72)};
\node (n3)  [mynode,join]   {Studies screened on title\\
                                (N = 72)};
\node (n4)  [mynode,join]   {Studies screened on abstract\\
                                (N = 31)};
\node (n5)  [mynode,join]   {Full-text accessed for eligibility\\
                                (N = 23)};
\node (n6)  [mynode,join]   {Studies included in qualitative synthesis\\
                                (N = 18)};
% \node (n7)  [mynode,join]   {\# of studies included in quantitative sysntesis\\
%                                (meta-analysis)};
% the branches to the right
\node (n3r) [mynode,right=of n3]    {Studies excluded:\\
                                        20: Not related to MW\\
                                        15: Related to impact of MW\\
                                        6: Related to attention guidance
                                        };
\node (n4r) [mynode,right=of n4]    {Studies excluded:\\
                                    2: Not related MW\\
                                    4: Related to impact of MW\\
                                    2: Related to attention guidance
                                    
                                        };
\node (n5r) [mynode,right=of n5]    {Studies excluded:\\
                                        2: Exclusive use of manual detection of MW\\
                                        2: Overview of the subject\\
                                        1: No focus on detection of MW};
% lines not included in join                                        
\draw[arrow] ([xshift=+22mm] n1a.south) coordinate (a)
                                       -- (a |- n2.north);
\draw[arrow] ([xshift=-22mm] n1b.south) coordinate (b)
                                       -- (b |- n2.north);
\draw[arrow] (n3) -- (n3r);
\draw[arrow] (n4) -- (n4r);
\draw[arrow] (n5) -- (n5r);
% the labels on the left
%    \begin{scope}[node distance=5mm]
%\node[mylabel,below left=-3mm and 5mm of n1a.north west]
%                {\rotatebox{90}{Identification}};
%\node[mylabel, minimum height=40mm]  {\rotatebox{90}{Screening}};
%\node[mylabel]  {\rotatebox{90}{Eligibility}};
%\node[mylabel]  {\rotatebox{90}{Included}};
%    \end{scope}
\end{tikzpicture}}
\caption{Flow Diagram outlining the screening process}
\label{fig:prisma}
\end{figure}

Second, all duplicates obtained from the different libraries were removed, leaving 72 articles.

Third, from this point onward, all remaining articles were divided into two stacks. Each stack was reviewed individually by two team members.
The articles were screened on their title and were excluded if team members reached consensus on their irrelevancy. 
This resulted in three exclusion criteria:
\begin{enumerate}
    \item \textbf{Not related to Mind Wandering}, e.g. \cite{ISI:000432512400017}, focuses on different formats to present lecture material and is therefore excluded.
    \item \textbf{Related to the impact of Mind Wandering}, e.g. \cite{Albert2018LinkingDrivers}, concludes that mind wandering while driving leads to dangerous behaviour behind the wheel and is thus excluded.
    \item \textbf{Related to attention guidance}, e.g. \cite{Xiao2018ClassroomMechanism}, explores gamification to restore students' attention level in classroom teaching and is thus excluded.
  \end{enumerate}
In case of uncertainty, articles were included to be screened on abstract. This step narrowed the list of articles down to 31.

Fourth, Eight articles were excluded while screening on abstract, leaving 23 articles remaining. 
The excluded articles fit the earlier defined exclusion criteria but looked interesting based on their title.

Fifth, every remaining article was accessed and screened on their full-text. This included reading the methodology, results, conclusion and inspecting their figures and tables.
Two papers were excluded because they give an overview of the field, rather than explaining (new) methods of detecting mind wandering. 
Another two articles were excluded because they use manual reporting exclusively.

At last, one article was excluded because it describes a system that uses third-party solutions to detect mind wandering. The focus of this article lays on implementing this system in a learning environment, rather than on combining/improving detection methods for mind wandering.
In conclusion, a total of 18 articles are left to review.

Finally, the articles were carefully studied, extracting data of the methods used to detect mind wandering and storing these data in a table. This table is then further analyzed.
