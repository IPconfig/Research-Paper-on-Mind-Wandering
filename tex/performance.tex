Different ML algorithms were used to train models that can detect MW. Often, many ML algorithms are used for the classification tasks because it is not yet known which algorithm is best suited to train a model that can automatically detect MW. Often, the ML algorithms that are used are from Weka\footnote{https://www.cs.waikato.ac.nz/ml/weka/} \cite{Bixler2015AutomaticPhysiology}\cite{Bixler2015AutomaticAwareness}\cite{Bixler2016AutomaticReadingd}\cite{Bixler2014TowardWanderingd}\cite{Blanchard2014AutomatedLearning}\cite{Gwizdka2019ExploringTasks}\cite{Hutt2017OutClassroom}\cite{Pham2015Attentivelearner:Tracking}, a collection of machine learning algorithms. From these algorithm(s) the best one(s) would be picked. Most studies concluded that Support Vector Machines and Bayesian models performed best. There were also studies that used no ML algorithms. As is the case for \cite{Jo2017AMind}, in this study, initial evidence was provided that high-frequency words in lecture videos could be used to detect MW. It also occurred that ML could not be applied, this happened in \cite{Gontier2016HowEnvironment} and was due to the fact that there was only one participant in the study. This resulted in a too small amount of data to train the model with.

The performance was measured in different ways. The performance was measured using a F$_1$ score, Cohen's kappa or as accuracy. Accuracy can be defined in different ways, mostly dependent on the application. For example, in \cite{Cheetham2016AutomatedApplication}, accuracy was defined as the area under the receiver operator characteristic curve, while in \cite{Mishchenko2015DetectingTespiti} cross-validation accuracy is used. This makes it hard to compare different studies. Some properties were however also compared within the study. In \cite{Gwizdka2019ExploringTasks}, the results show that there is no significant difference in accuracy between a window size of 5 seconds and 10 seconds. This confirms that the optimal window size is not the same for every application and study. This study also showed a significant increase in accuracy over what was achieved in \cite{Bixler2014TowardWanderingd}. As mentioned before, it is important to verify whether MW could also be detected by using equipment that is owned by more people. One study found that a low-end webcam performed as good as an eye tracker \cite{Zhao2017ScalableApproach}. Yet another study achieved an accuracy of 53\% using a mobile device \cite{ISI:000443429900018}, showing a slight decrease in accuracy compared to similar studies using an eye tracker.