The biggest problem with detecting mind wandering is that it is an internal state of the mind and is thus something that can not easily be measured. This makes the process of collecting training data quite difficult, because you have to rely on alternate methods to gather the information of whether a person is mind wandering.

In most papers, this was done with some type of probe that was activated while the test subject was performing the tasks mentioned in Section \ref{sec:tasks}. A probe in this context means a signal to the test subject, who then records whether he/she was mind wandering. This is a simple binary yes/no answer, no intensity of the MW is given in any of the studies. In \cite{Bixler2015AutomaticPhysiology}, it is stated that probes are the most standard way for reporting MW, because alternatives like EEG and fMRI have not been validated and are often not practical. The most common type of probes were within-page and end-of-page probes (only applicable when the task was reading) and auditory probes, which give a certain sound when the test subject has to report whether or not they were mind wandering.

Another technique for reporting MW that was commonly used is self-caught reports (SCR). With this technique, the test subject has to report the MW the moment he/she becomes aware of it, which means they report a form of MW that occurs with metacognitive awareness \cite{Bixler2015AutomaticAwareness}. An obvious downside to SCR is that there is no information about the time where no reports of MW occur. The test subject could be paying attention, but could also be mind wandering without realizing it yet \cite{Bixler2015AutomaticAwareness}. This is a problem that does not occur when using probes, since the signal indicates when the subject reports, thus always guaranteeing a yes or no answer. Although this is a clear limitation of SCR, both the probe-caught and self-caught methods have been validated in a number of studies \cite{Bixler2015AutomaticPhysiology}. 

In terms of performance, both the different types of probes and SCR showed success in different studies, shown in Table \ref{tab:data}, making both of them viable options to construct training data. 

Another thing that is important when gathering data about MW is how much time you record of the MW phase, this is called the window size. Not every study mentioned the window sizes used and the ones that did often had varying different windows, anywhere from 1 to 60 seconds. Some studies also tried the same experiment with various window sizes. Zhang stated that they ran their experiment with window sizes 2, 8 and 16, and for all machine learning models, a window size of 8 gave the highest accuracy \cite{ISI:000443429900018}. Overall, the optimal window size is probably very dependent on the study, so a conclusion about the best general window size to detect mind wandering can not be drawn.