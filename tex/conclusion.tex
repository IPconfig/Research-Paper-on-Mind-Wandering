After doing the literature study, a few interesting remarks can be made regarding existing and possible future research. 

Firstly, the lack of published datasets is a remarkable point. None of the reviewed literature published their datasets or have used publicly available datasets. Having these datasets published would be beneficial for future researchers since they do not have to spend precious time collecting data before engineering their algorithm. This will also enable researchers to compare their performance with results obtained in previous studies.

Secondly, all experiments were done in a laboratory with people that knew that their mind wandering was going to be tracked. This could result in people being more focused than they would normally be because they know their performance is being measured and because they are in this experimental setting. It may be interesting to do some research in a more casual environment, although gathering data this way could be harder because probes could not be used.

Lastly, comparing the different studies and technologies was quite difficult because a lot of different performance measures were used. Some mentioned accuracy, some used an F$_1$ score and others used Cohen's kappa measure. In order to be able to compare the performance of different technologies, it would be beneficial if all studies measured performance in the same way. 

Overall, there is a lot of potential in this field of research in terms of real-world applications. The main focus here should be to have a high accuracy with as few \emph{false negatives} as possible. In real-world applications such as driving, it is a mild annoyance to have a false positive, but having a false negative can cause a very dangerous situation. The technology and research regarding the automatic detection of MW will hopefully grow in the coming years so it can be implemented to improve the efficiency of everyday tasks.