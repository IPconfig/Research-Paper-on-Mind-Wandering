After doing the literature study, a few interesting remarks can be made regarding existing research.

Firstly, the lack of published datasets is a remarkable point. With the exception of \cite{Zhao2017ScalableApproach}, none of the reviewed literature published their datasets and none of them used publicly available datasets. Having published datasets is of great importance for future research. This would enable researchers to use the same dataset and compare their results with results obtained in previous studies, giving them some additional context about their performance.

Secondly, all experiments were done in a laboratory with people that knew that their mind wandering was going to be tracked. This could result in people being more focused than they would normally be because they know their performance is being measured and because they are in this experimental setting. It may be interesting to do some research in a more casual environment, although gathering data this way could be harder because probes could not be used.

Lastly, comparing the different studies and technologies was quite difficult because a lot of different performance measures were used. Some mentioned accuracy, some used an F$_1$ score and others used Cohen's kappa measure. In order to be able to compare the performance of different technologies, it would be beneficial if all studies measured performance in the same way.  

Overall, there is a lot of potential in this field of research in terms of real-world applications. To improve future research, datasets have to be made publicly available and some of these datasets and reporting methods should be standardized in a way that comparisons can be made easily. The technology and research regarding the automatic detection of MW will hopefully grow in the coming years such that it can be implemented to improve the efficiency of everyday tasks.