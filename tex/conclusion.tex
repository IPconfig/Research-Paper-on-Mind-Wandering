After doing the literature study, a few interesting remarks can be made regarding existing and possible future research. 

Firstly, the lack of published datasets is a remarkable point. None of the reviewed literature mentioned anything about publishing the datasets they used or published. Publishing the datasets can be helpful for new researches that are being done because these researchers do not have to create their own datasets. Also, the researches that are already out there can be validated with the datasets and other ML algorithms could be used and the results could be compared. For future researches it would be more beneficial if the researchers publish the datasets they used or created. 

Secondly, most, if not all, experiments were done with people that knew that their mind wandering was going to be tracked. This could result in people being more focused than they would normally be when performing the same task, because they are in this experimental setting. It could possibly be interesting to do some research where people are in more casual environments, although gathering data this way could be harder because probes and such could not be used.

Lastly, comparing the different studies and technologies was quite difficult because a lot of different performance measures were used. Some mentioned accuracy, some used an F$_1$ score and others used Cohen's kappa measure. In order to be able to compare the performance of different technologies, it would be beneficial if all studies measured performance in the same way. 

Overall, there is a lot of potential in this field of research in terms of real-world applications. The main focus here should be to have a high accuracy with as few false negatives as possible. In real-world applications such as driving, it is a mild annoyance to have a false positive, but having a false negative can cause a very dangerous situation. The technology and research regarding the automatic detection of MW will hopefully grow in the coming years so it can be implemented to improve the efficiency of every day tasks.